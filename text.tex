\documentclass{article}
\usepackage{graphicx} % Required for inserting images
\usepackage{mathtools}

\title{DD1337 läxa}
\author{Arvid Kristoffersson}
\date{November 2024}

\begin{document}

\maketitle

\section{Induktion}
\subsection{a)}
\[ \sum_{i=1}^{n}i^2 = \frac{n(n+1)(2n+1)}{6}\]
Vi visar att påståendet stämmer med hjälp av induktion:\\
Basfall: $n = 1$\\
$VL = \sum_{i=1}^1 i^2 = 1^2 = 1$ ~ $HL = \frac{1(1+1)(2\cdot1+1)}{6} = \frac{6}{6} = 1$\\
$VL = HL$ Basfallet stämmer alltså.\\
Anta att påståendet stämmer för alla $k$ mindre eller lika med något $p$ där $k, p \in \mathbf{Z}_+$.\\
Vi visar nu för $p+1$:\\
\[ VL = \sum_{i=1}^{p+1}i^2 = \sum_{i=1}^p i^2 + (p+1)^2\] Vilket enligt antagande ger:
\[ VL = \sum_{i=1}^pi^2 + (p+1)^2 = \frac{p(p+1)(2p+1)}{6} + (p+1)^2 = \frac{2p^3+p^2+2p^2+p}{6} + (p+1)^2 = \frac{2p^3+3p^2+p}{6} + p^2+2p+1 \] 
\[= \frac{2p^3+3p^2+p}{6} + \frac{6p^2+12p+6}{6} = \frac{2p^3+9p^2+13p+6}{6}\]
\[HL = \frac{(p+1)(p+1+1)(2(p+1)+1)}{6} = \frac{(p+1)(p+2)(2p+3)}{6} = \frac{2p^3+9p^2+13p+6}{6}\]
\[VL = HL\]
Att påståendet stämmer för alla $n \in \mathbf{N}$ följes av induktionsprincipen.\\
\textbf{V.S.B}

\newpage

\subsection{b)}
Vi ska visa att följande påstående gäller.
\[
    \sum_{i = 1}^{n}(2i-1)=n^2
\]

Vi visar detta med hjälp av induktion:\\
Basfall: $n = 1$ \\
$ VL = \sum_{i=1}^{1}(2i-1) = 2\cdot1-1 = 1$\\
$ HL = 1^2 = 1$\\
$ VL = HL$ Påståendet stämmer alltså för basfallet\\
Anta att påståendet stämmer för alla $k$ mindre eller lika med något $p$, för $p, k \in \mathbf{Z}_+$.\\
Vi visar för $p+1$:\\
\[ VL = \sum_{i=1}^{p+1}(2i-1) = \sum_{i=1}^p(2i-1)+2(p+1)-1\]
Vilket enligt antagande blir:\\
\[ = p^2+2(p+1)-1 = p^2+2p+1\]\\
\[ HL = (p+1)^2 = p^2+2p+1 \] Alltså är:\\
\[ VL = HL\]
Att påståendet stämmer för alla $p \in \mathbf{Z}_+$ följes av induktionsprincipen.\\
\textbf{V.S.B}

\newpage

\section{Iterativ korrekthet}
Vi bevisar dens korrekthet med induktion.\\
Vi vill visa att expIterative(double x, int n) returnerar $x^n$ för alla $n \in \mathbf{N}$ och $x \in \mathbf{Q}$ eftersom $x$ är en double.\\
Basfall: $n = 0$\\
expIterative(x, 1) ger att for-loopen aldrig körs och res initiala värde returneras, alltså $1.0$. Vilket stämmer, eftersom $x^0 = 1$.
\\
Anta nu att påståendet stämmer för varje $k$ mindre eller lika med något $p$, där $k, p \in \mathbf{N}$. Vi visar nu för $p+1$.\\ 
Låt först fr(0, u) $\longrightarrow$ f(x) beteckna en for loop som går från 0 till u och kör någon funktion f(x) där x är vilket steg den är på (0...u).\\
\[VL = fr(0, p+1) \longrightarrow res =res\cdot x = (fr(0, p) \longrightarrow res = res\cdot x)\cdot x \] \\
Vilket enligt antagande blir:\\
\[ = (x^p)\cdot x = x^{p+1}\]
\[ HL = x^{p+1}\]\\
\[VL = HL\]
Att påståendet stämmer för alla $n \in \mathbf{N}$ följes av induktionsprincipen.\\ \\
Tidskomplexiteten är $O(N)$ (alltså linjär) för att den gör en $O(1)$ operation $N$ gånger.

\newpage

\section{Rekursiv korrekthet}
\subsection{korrekthet}
Vi visar även denna med induktion. Vi vill visa att expRecursive(double x, int n) returnerar $x^n$ för alla $n \in \mathbf{N}$ och $x \in \mathbf{Q}$.\\
Basfall: $n= 0, n = 1, n=2, n=3, n=4$ 
När $n$ antar något av dessa värden returnerar funktionen expIterative(x, n) vilket vi redan har bevisat stämmer för alla $n \in \mathbf{N}$. Alltså stämmer det även för dessa basfall.\\
Antag att påståendet att expRecursive(x, k) returnerar $x^k$ för alla $k$ mindre eller lika med något $4 \leq p \in N$. Vi visar nu för $p+1$:\\
\[ VL = expRecursive(x, p+1) = expRecursive(x, \lfloor{(p+2)/2 \rfloor}) \cdot expRecursive(x, \lfloor(p+1)/2\rfloor)\] Om $p$ är jämnt får vi: \[
VL = expRecursive(x, p/2+1)\cdot expRecursive(x, p/2)
\] 
Detta blir enligt antagande (notera att $p/2+1$ och $p/2$ är mindre än $p+1$ för alla $p \in \mathbf{N}$:\\

\[ expRecursive(x, p/2+1)\cdot expRecursive(x, p/2) = x^{p/2+1}\cdot x^{p/2} = x^{p/2+1+p/2} = x^{p+1}\]
och om $p$ är udda får vi: \[
VL = expRecursive(x, (p+1)/2)\cdot expRecursive(x, (p+1)/2)
\]
Vilket enligt antagande blir: \\
\[
expRecursive(x, (p+1)/2)\cdot expRecursive(x, (p+1)/2) = x^{(p+1)/2}\cdot x^{(p+1)/2} = x^{(p+1)/2 + (p+1)/2} = x^{p+1}
\]

\[ HL = x^{p+1}\]
\[ VL = HL \] stämmer för både jämna och udda $p$.\\
Att påståendet stämmer för alla $n \in \mathbf{N}$ följes av induktionsprincipen.
\textbf{V.S.B}

\subsection{tidskomplexitet}
Mästarsatstermerna a, b och f(n) är i detta fall $a = 2, b=2, f(n)=O(1)$. Då får vi fallet i mästarsatsen där $T(n) = O(n^{log_b(a)}) = O(n^{log_2(2)}) = O(n)$\\
Tidskomplexiteten är alltså linjär ($O(n)$).\\

Man kan också bevisa det mer intuitivt. Fallet när $n <= 4$ antar vi vara konstant eftersom det maximalt blir $4$ operationer. Tänk dig att funktionen bildar ett träd där varje nod har två barn vars värden är hälften av nodens, ända tills noden uppfyller $n <= 4$. Notera att ett värde $n$ kan halveras maximalt logaritmiskt antal gånger bas 2. Detta innebär att trädet får en höjd som är $log_2(n)$. Vi kan se det som att trädet har logaritmiskt antal lager. Notera att antalet noder i varje lager dubbleras varje gång också. Eftersom varje tidigare nod bildar har två barn. Alltså blir antalet noder i lager $p$ (räknat från toppen till botten från $0 \longrightarrow log_2(n)$) $2^{p}$. Eftersom varje nod genomför konstant antal operationer ($O(1)$) blir komplexiteten samma som antalet noder. Tänk dig nu att vi summerar alla $2^p$: alltså $\sum_{p=0}^{log_2(n)}2^p$. Man kan tänka på detta som summan av alla tal $\leq p$ som har endast en bit i sin binära representation. Detta binära  tal kommer att vara på formen $11...1$ med $log_2(n)$ antal bitar. Värdet av detta binära tal blir därför $2\cdot n -1$ vilket är hur många noder det finns ungefär. Tidskomplexiteten blir alltså $O(n)$.



\end{document}
